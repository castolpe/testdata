\documentclass[10pt,a4paper]{article}
\usepackage[latin1]{inputenc}
\usepackage[ngerman]{babel}
\usepackage{amsmath}
\usepackage{amsfonts}
\usepackage{amssymb}
%\parindent0pt 

\title{Testdata}

\begin{document}

\maketitle

\section{�bersicht Haushalte}

\subsection*{Haushalt 1}

Deutsche Familie mit 2 Kindern unter 16 Jahren, 2 Kindern �ber 16 Jahren (Sch�ler 18 Jahre,
Student 19 Jahre), wohnhaft in reinem Wohngebiet mit �berwiegend Altbauten, Miete
1025 Euro, HH-Nettoeinkommen 4573 Euro. Eltern verheiratet und
zusammenlebend

\begin{itemize}
  \item Vater (34 Jahre) Polizeimeister, im Ausland geboren, vor 1989 in BRD lebend
  \item Mutter (32 Jahre) Lehrerin, vor 1989 in DDR lebend
\end{itemize}

\subsection*{Haushalt 2}

Franz�sischer Student, 21 Jahre,  ledig, ohne Kinder, 2000 nach Deutschland gezogen,
wohnhaft in einem Mischgebiet mit Wohnungen und Gesch�ften bzw. Gewerbebetrieben. HH-Nettoeinkommen 850 Euro, Miete 300 Euro

\subsection*{Haushalt 3}

Deutsche Einzelhandelskauffrau, 26 Jahre, ohne Kinder, wohnhaft in einem reinen Wohngebiet mit �berwiegend Neubauten, HH-Nettoeinkommen 1150 Euro, Miete 390 Euro

\subsection*{Haushalt 4}

Deutsch (Mutter, Hausfrau, 30 Jahre)-t�rkische (Vater, 40 Jahre, Hotelmanager, seit 1981 in Deutschland als Asylbewerber od. Fl�chtling) Familie mit 3 Kindern (unter 16 Jahre), verheiratet und zusammenlebend, wohnhaft in reinem Wohngebiet mit �berwiegend Altbauten, HH-Nettoeinkommen 3458, Miete 950

\subsection*{Haushalt 5}

Polnische verwitwete Frau, 28 Jahre, 1 Kind, Designerin,  seit 1980 in BRD, wohnhaft in einem Gesch�ftszentrum mit wenigen Wohnungen, HH-Nettoeinkommen 2984, Miete 900

\subsection*{Haushalt 6}

26 j�hriger deutscher Softwareentwickler, ledig, ohne Kinder, geboren in DDR, wohnhaft in einem Mischgebiet mit Wohnungen und Gesch�ften bzw. Gewerbebetrieben, Miete 480, HH-Nettoeinkommen 1600

\subsection*{Haushalt 7}

Geschiedener deutscher Mann(43 Jahre) mit einem Kind unter 16 Jahren, Ingenieur,  wohnhaft in reinem Wohngebiet mit �berwiegend Altbauten, vor 1989 in DDR

\subsection*{Haushalt 8}

Deutsche Familie mit 4 Kindern unter 16 Jahren, verheiratet 

\begin{itemize}
  \item Mutter Buchhalterin, 46 Jahre, im Ausland geboren, seit 1970 in BRD lebend;
  \item Vater Professor, 45 Jahre, wohnhaft in reinem Wohngebiet mit �berwiegend Neubauten, Miete 1000, HH-Nettoeinkommen 7413
\end{itemize}

\subsection*{Haushalt 9}

Deutsches Paar ohne Kinder

\begin{itemize}
  \item Frau 18 Jahre, ledig, Reinigungskraft, vor 1989 in DDR wohnhaft
  \item Mann 21 Jahre, ledig, arbeitslos, vor 1989 in DDR wohnhaft, Fachschule Abschluss
  \item �nderung in 2002: Arbeit als Augenoptiker, HH-Nettoeinkommen �ndert sich auf 2060 Euro
\end{itemize}

\subsection*{Haushalt 10}

\begin{itemize}
  \item Deutscher Mann, 32 Jahre, geschieden, mit Ex-Frau zusammenlebend, Chemietechniker
  \item Schwedische Frau, 27 Jahre, geschieden, mit Ex-Mann zusammenlebend, Kinderg�rtnerin
  \item 1 gemeinsames Kind unter 16 Jahre, wohnhaft in einem Gewerbe- bzw. Industriegebiet mit wenigen Wohnungen, Miete 800, HH-Nettoeinkommen 5194, beide vor 1989 in BRD wohnhaft
  \item Oma, 75 Jahre, deutsch, verwitwet, seit 1936 in Deutschland wohnhaft, Aussiedler (d.h. deutschst�mmige Person aus osteurop�ischen Staaten), vor 1989 in BRD wohnhaft
  \item �nderung 2002: Kind kommt zur Welt
  \item �nderung 2003: Oma verstirbt
\end{itemize}

\subsection*{Haushalt 11}

89 j�hriger Rentner, deutsch, verwitwet, pers. Bruttoeinkommen 1400, Miete keine Kinder, wohnhaft im reinen Wohngebiet mit �berwiegend Neubauten, vor 1989 in BRD wohnhaft, keine Angabe zur Miete

\subsection*{Haushalt 12}

Vietnamesischer Koch, 37 Jahre, ledig, ohne Kinder, seit 1990 in Deutschland wohnhaft, HH-Nettoeinkommen 1780, Miete 650, wohnhaft im reinen Wohngebiet mit �berwiegend Altbauten

\begin{itemize}
  \item �nderung 2002: verheiratet, aber getrennt lebend mit deutscher Frau (32 Jahre), Friseurin
  \item �nderung 2003: Wohnungswechsel, reines Wohngebiet mit �berwiegend Neubauten
\end{itemize}

\end{document}